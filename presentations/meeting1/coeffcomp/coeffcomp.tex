\documentclass[aspectratio=169]{beamer}

\usepackage{amsmath}
\usepackage{hyperref}

\title{Comparing The Estimated Discrete-Time Model From PNLSS With The
  Original Continuous-Time Model}
\subtitle{Project 2 WP2: TRC 2019}
\author{Meeting 1}
\date{Fri Jun 28, 2019}

\begin{document}
\maketitle{}

\begin{frame}[allowframebreaks]
  \frametitle{Linear Terms}
  \begin{itemize}
  \item PNLSS is implemented in discrete time and hence, finding
    physical meanings for the obtained coefficients is not trivial
  \item For the linear parts of the system, the conversion may be
    carried out using the state transition matrix\\
    
    \begin{columns}
      \begin{column}{0.5\linewidth}
        \centering
        \textbf{Continuous Time Model}
        \begin{align*}
          \dot{X} &= \mathbf{A}X + \mathbf{B}u\\
          Y &= \mathbf{C}X + \mathbf{D}u
        \end{align*}
      \end{column}
      \begin{column}{0.5\linewidth}
        \centering
        \textbf{Discrete Time Model}
        \begin{align*}
          \dot{X}_{k+1} &= \mathbf{A_d}X_k + \mathbf{B_d}u_k\\
          Y_k &= \mathbf{C_d}X_k + \mathbf{D_d}u_k
        \end{align*}
      \end{column}    
    \end{columns}
    
    \textbf{Conversion Formulae:}
    \begin{alignat*}{8}
      \mathbf{A_d} &= e^{\mathbf{A}\Delta t}\qquad &\mathbf{B_d} &=
      \mathbf{A}^{-1}\left(\mathbf{A_d}-\mathbf{I}\right)
      \mathbf{B}\qquad &\mathbf{C_d} &= \mathbf{C}\qquad &\mathbf{D_d} &=
      \mathbf{D}\\
      \mathbf{A} &= \log\left(\mathbf{A}\right)f_{samp}\qquad &\mathbf{B} &=
      {\left(\mathbf{A_d}-\mathbf{I}\right)}^{-1} \mathbf{A} \mathbf{B_d}
      \mathbf{B}\qquad &\mathbf{C} &= \mathbf{C_d}\qquad &\mathbf{D} &=
      \mathbf{D_d}\\
      && f_{samp} &= \frac{1}{\Delta t}
    \end{alignat*}
  \item The model identified from PNLSS need not have states identical
    to the physical ones used for simulations (displacements,
    velocities). 
  \item So these matrices must be transformed to a canonical form so
    that the coefficients may directly be compared
  \item The physical transformation based on (Etienne Gourc, JP Noel,
    et.al "Obtaining Nonlinear Frequency Responses from Broadband
    Testing"
    \url{https://orbi.uliege.be/bitstream/2268/190671/1/294_gou.pdf})
    \begin{columns}
      \begin{column}{0.5\linewidth}
        \begin{alignat*}{4}
          \text{Original Model} \begin{bmatrix}
            0 & 1\\ -2.7665\times 10^6 & -12.6409
          \end{bmatrix}, &\begin{bmatrix} 0\\
            1.3195 \end{bmatrix}, &\begin{bmatrix} 1 &
            0 \end{bmatrix}, &\begin{bmatrix} 0 \end{bmatrix}\\
          \text{PNLSS:} F_{RMS}=15 \begin{bmatrix}
            0 & 1\\ -2.7676\times 10^6 & -12.7202
          \end{bmatrix}, &\begin{bmatrix} 2.0402\times 10^{-4}\\
            1.2888 \end{bmatrix}, &\begin{bmatrix} 1 &
            0 \end{bmatrix}, &\begin{bmatrix} 0 \end{bmatrix}\\
          \text{PNLSS:} F_{RMS}=150 \begin{bmatrix}
            0 & 1\\ -2.7676\times 10^6 & -12.7251
          \end{bmatrix}, &\begin{bmatrix} 2.0394\times 10^{-4}\\
            1.2886 \end{bmatrix}, &\begin{bmatrix} 1 &
            0 \end{bmatrix}, &\begin{bmatrix} 0 \end{bmatrix}
        \end{alignat*}
      \end{column}
    \end{columns}
  \end{itemize}
\end{frame}

\begin{frame}
  \frametitle{Non-Linear Terms}
  \begin{itemize}
  \item Things are a little more complicated for the nonlinear terms
  \item We're unable to find a proper way to transform the discrete
    time coefficient matrix (\textbf{E} in the code) to a continuous
    time equivalent. 
  \item Maybe integrals are involved? Since we have time history data
    and also the nonlinearities are smooth, we may be able to
    evaluate the integrals accurately.
  \item We have tried to convert the coefficients matrices to the
    physical domain (as in the previous slide), but are unable to
    proceed further in order to truly compare the coefficients.
  \end{itemize}
\end{frame}

\end{document}
%%% Local Variables:
%%% mode: latex
%%% TeX-master: t
%%% End:
